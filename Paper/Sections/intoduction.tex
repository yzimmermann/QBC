Historically, the Keynesian Beauty Contest is a game that describes the interaction of rational agents, in a context where they are asked to choose the most attractive person among a group of people. The players who picked the most popular face are then rewarded with a prize.\\

The catch of this game is that the player should not really pick the most attractive person in accordance to their personal opinion, but should rather be aware of the opinions of the other agents in order to formulate a decision. This line of thought can be carried recursively, since the player can then formulate their decision based on their expected opinion of the other player's public perception of beauty, and so on.\\

In the words of John Maynard Keynes himself:
\begin{displayquote}
``...it is not a case of choosing those [faces] that, to the best of one's judgment, are really the prettiest, nor even those that average opinion genuinely thinks the prettiest. We have reached the third degree where we devote our intelligences to anticipating what average opinion expects the average opinion to be. And there are some, I believe, who practice the fourth, fifth and higher degrees." \citep{Keynes_1936}
\end{displayquote}
Also, as Keynes remarks in this work, the Beauty Contest has strong parallels with the behavior of professional investors in financial markets, in the sense that investors try to be one step ahead of the average investor, which is seen as an ignorant agent that is likely bound to be present in the market \citep{Duffy1997beauty}.\\

All the features that characterize this historical version of the Beauty Contest can be elegantly put in a mathematical framework which formalizes the setup. In the following section, we will firstly discuss this classical version of the game, which is a standard introductory example in Game Theory courses. We will then devote our attention to the formulation, computational implementation and analysis of a quantum version of the game.\\