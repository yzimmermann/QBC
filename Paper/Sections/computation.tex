The strategies of the Quantum Beauty Contest live in a high-dimensional Hilbert space. To effectively explore and optimize these strategies, an evolutionary algorithm is employed. The algorithm utilizes reproduction and mutation, which enables effective sampling of the strategy space to find optimal strategies. A similar algorithm has recently been used to computationally study the prisoner's dilemma \citep{vie2021evolutionary}.\\

\subsection{Reproduction}

Let $\ket{{\text{S}_1}}$ and $\ket{{\text{S}_2}}$ represent the quantum states of two parent strategies. The reproduction operator $$R: \mathcal{H} \times \mathcal{H} \rightarrow \mathcal{H}$$ combines these states to generate a new strategy for the offspring:
\begin{equation}
\ket{\text{S}_{\text{offspring}}} = {R}\left(\ket{\text{S}_1}, \ket{{\text{S}_2}}\right) = \mathcal{N}(\ket{\text{S}_1} + \ket{{\text{S}_2}})
\end{equation}

This process ensures that the offspring inherits characteristics from both parents while exploring new regions of the strategy space.\\

\subsection{Mutation}

The mutation operator introduces variation by perturbing individual strategies after reproduction.\\

Let's denote the mutation operator as $M(\ket{S}, \pi, \sigma)$, where $\ket{S}$ is some strategy, $\pi$ is the mutation rate, and $\sigma$ is the mutation strength. The mutation operator can be defined as follows:
\begin{align}
&M(\ket{S}, \pi, \sigma) = \ket{S'}\\
&\ket{S'}_i = \left\{
\begin{array}{ll}
\ket{S}_i + \delta & \text{with probability } \pi \\
\ket{S}_i & \text{with probability } 1 - \pi
\end{array}
\right.
\end{align}

where $\ket{S'}$ represents the mutated strategy, and $\delta$ is a random perturbation drawn from a uniform distribution $\mathcal{U}(\left[-\sigma, \sigma\right])$. If a mutation event occurs (with probability $\pi$), the corresponding component of the strategy is perturbed by adding a random value from $\delta$. Therefore, a higher mutation rate leads to more frequent mutations and greater exploration of the strategy space, while a lower mutation rate promotes exploitation of existing elite strategies.\\

On the other hand, a larger mutation strength results in larger deviations from the original strategy, allowing for more significant exploration of the strategy space. Conversely, a smaller mutation strength limits the extent of perturbations, promoting finer adjustments to the strategies.\\

\subsection{Algorithm}

Here is a quick overview of the algorithm employed:
\begin{enumerate}
    \item \textbf{Initialize:} For every player i, draw random strategy ket $\ket{S_i} \sim \mathcal{U}_{101}(\left[-1, 1\right]^{101})$ with $\braket{S_i|S_i}=1$    
    \item \textbf{Simulate Game:} 
    \subitem \hspace{-0.7cm}a. Draw $\ket{n}$ from $|\braket{\psi|\psi}|^2$ as defined in Eq. \ref{eq:game_ket} and determine winner ket $\ket{w}$
    \subitem \hspace{-0.7cm}b. Determine fitness of $\ket{\text{S}_i}$ with respect to $\ket{w}$ for every strategy
    \item \textbf{Reproduction Cycle}
    \subitem \hspace{-0.7cm}a. Extract fixed amount of elite strategies into new game
    \subitem \hspace{-0.7cm}b. Select parent strategies $\ket{\text{S}_i}$, $\ket{\text{S}_j}$  $i \neq j$ and form new strategy $M(R(\ket{\text{S}_i}$, $\ket{\text{S}_j}), \pi, \sigma)$
    \subitem \hspace{-0.7cm}c. Repeat b. until every player has a strategy
    \item \textbf{Iterate:} Jump to 2. until convergence or fixed number of iterations reached
\end{enumerate}

For the specific implementation of the game we can make different choices for how we determine $\ket{w}$ as well as the fitness. The implementations can be found in the class functions \texttt{fitness(self)} and \texttt{measurement(self)}, respectively. The full source code can be found in Appendix \ref{apd:first}.\\


%(PLOTS OF RUNS)\\
%(INTERPRETATION)\\




