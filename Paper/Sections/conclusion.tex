With the Quantum Beauty Contest we were able to apply a formalism known from the theory of quantum mechanics onto the classical game of the Keynesian beauty contest. In particular, we looked at the case of a contractive factor $p = 1/2$ but other cases could also be easily investigated in the future.\\

With different approaches on how to determine the fitness of the strategy of a specific player we were able to observe and describe several phenomena in an evolutionary setting which are not predicted by the classical game. With phenomena such as a phase transition on the projective game approach we are able to draw connections to human behavior, in this particular example the transition from confusion to clarity on the interactions within the game. Still it is uncertain if these simulations could ever be closely reproduced by humans, considering the relatively simple mutation algorithm we used throughout the simulations. Here one could perform further investigations by limiting the amount of vector components one strategy can contain, changing the adaptation from parent strategies or limiting the precision. These might be able to closer represent a human game where people would probably only use simple fractions and not choose $101$ states for many rounds.\\

By using the principle of wave function collapse we were also able to bring some instability to the game in its equilibrium position forcing older strategies to adapt to different conditions. While strategies still show dominant $\ket{0}$ states, using the player average does not allow them to fully neglect other states. Only by using the projection operation a player is able to largely focus on this state when at the Nash equilibrium.\\

With the approach of taking the weight of the average we showed another interesting result as the game did converge but always to a state higher than $\ket{0}$. With humans it might therefore be interesting if we can reproduce such a behavior and especially to which points humans would converge. Here using smaller values for the contraction factor such as $p = 1/3$ might also be interesting to investigate in the future as this currently sets a hard boundary at $\ket{1}$ and makes a measurement of $\ket{0}$ impossible.\\

One last thing which was not considered yet by us is a variation in initial conditions. While we chose the initial states at random, leaving us with a relative uniform distribution in states, the classical game shows that most players tend to avoid numbers that are impossible to be determined as the average already on the first move. Manipulating the initial states might therefore also a a point for investigation in the future.\\